\documentclass{article}

\title{Learning neural network-based boundary conditions for kinetic plasma sheath dynamics}

\usepackage{graphicx}
\usepackage{amsmath}
\usepackage{amssymb}
\usepackage[dvipsnames]{xcolor}
\usepackage{siunitx}
\usepackage[margin=1in]{geometry}
\usepackage{parskip}

\usepackage{biblatex}
\addbibresource{references.bib}

\newcommand{\jack}[1]{{\color{ForestGreen} #1}}

\begin{document}
\maketitle

\section{Normalized Vlasov-Fokker-Planck equation}

The governing kinetic equation for our study is the 1D1V Vlasov-Dougherty-Fokker-Planck equation in
the ``flexible plasma normalization'' \cite{millerMultispecies13momentModel2016}:
\begin{align}
    \label{eqn:vlasov_fp}
    \partial_t f_s + v \partial_x f_s + (\omega_c \tau) \frac{Z_s}{A_s} E \partial_v f_s = (\nu_p \tau) \sum_s \nu_{s s'} \partial_v \left( \frac{T_{s s'}}{m_s} \partial_v f_s + (v - u_{s s'}) f_s \right).
\end{align}
The normalization constants appearing in this equation are as follows:
\begin{itemize}
    \item $Z_s$ and $A_s$ are the normalized charge and mass of species $s$, expressed in units of the proton charge and mass.
    \item $\omega_c \tau$ is the normalized reference proton cyclotron frequency in a field $B_0$ which would give unit plasma beta: $|B_0|^2 / 2 \mu_0 = n_0 T_0$.
    \item $\nu_p \tau$ is the normalized proton collision frequency.
\end{itemize}
For now we can take these normalization constants as given; we will need to concern ourselves with their definitions when we
move on to translating a specific physical problem into our equation setup.

Equation \eqref{eqn:vlasov_fp} is coupled to the normalized Gauss's law,
\begin{align}
\partial_x E = \frac{(\omega_p \tau)^2}{\omega_c \tau} \rho_c,
\end{align}
where
\begin{align}
    \rho_c = \sum_s Z_s \int f_s \,\mathrm{d} v
\end{align}
is the charge density.
We will use the elliptic form of Gauss's law,
\begin{align}
\partial_x^2 \phi = -\frac{(\omega_p \tau)^2}{\omega_c \tau} \rho_c,
\end{align}
with $E = -\partial_x \phi$.

\subsection{Collision operator}

\jack{Suggest we use the collision parameters derived in \cite{habbershawNonlinearConservativeEntropic2024}}


\section{Domain and boundary conditions}

We'll use a physical domain of length $L_x$, which by convention extends from $-L_x/2$ to $L_x/2$.

\subsection{Absorbing wall}
The simplest boundary condition that will produce a Langmuir sheath is the absorbing wall
boundary condition. At a spatial boundary $x_b$ with outward normal vector $\mathbf{n}(x_b)$, we have
\begin{align}
    f_s^b(x_b, v) = \mathbf{1}_{\mathbf{n}(x_b) \cdot v < 0} f_s(x_b, v),
\end{align}
where $\mathbf{1}$ is the indicator function.


\section{Straightforward DLR approximation}

Each species distribution function $f_s$ receives a separate low-rank decomposition:
\begin{align}
    f_s \approx \sum_{ij} X_{si}(x, t) S_{sij}(t) V_{sj}(v, t).
\end{align}
In what follows we will omit species subscripts.

Given that we need to apply boundary conditions in $x$, which should be applied to $K_j(x, t)$ rather
than $X_i(x, t)$, we keep all spatial derivatives applied to $K$.

\textbf{K step:}
\begin{align*}
    &\partial_t K_j(x, t) + \sum_l \left\langle V_j, v V_l \right\rangle_v \partial_x K_l(x, t) + (\omega_c \tau) \frac{Z_s}{A_s} \sum_l \left\langle V_j, \partial_v V_l \right\rangle_v E K_l(x, t) \\
    &\quad = \nu_p \tau \sum_{s'} \nu_{ss'} \sum_l \left[ \frac{T_{s s'}}{m_s} \left\langle V_j \partial_v^2 V_l \right\rangle_v + \left\langle V_j, \partial_v (v V_l) \right\rangle_v - u_{ss'} \left\langle V_j, \partial_v V_l \right\rangle_v \right] K_l(x, t)
\end{align*}

\textbf{S step:}
\begin{align*}
&\partial_t S_{ij}(x, t) + \sum_{l} \left\langle V_j, v V_l \right\rangle_v \left\langle X_i, \partial_x K_l \right\rangle_x + (\omega_c \tau) \frac{Z_s}{A_s} \sum_{kl} \left\langle V_j, \partial_v V_l \right\rangle_v \left\langle X_i, E X_k \right\rangle_x S_{kl}(t) \\
&\quad = \nu_p \tau \sum_{s'} \nu_{ss'} \sum_{kl} \left[ \frac{1}{m_s}\left\langle X_i, T_{ss'} X_k \right\rangle_x \left\langle V_j, \partial_v^2 V_l \right\rangle_v + \delta_{ik} \left\langle V_j, \partial_v (v V_l) \right\rangle_v - \left\langle X_i, u_{ss'} X_k \right\rangle_x \left\langle V_j, \partial_v V_l \right\rangle_v \right] S_{kl}(x, t)
\end{align*}

\textbf{L step:}
\begin{align*}
&\partial_t L_i(v, t) + \sum_k \left\langle X_i, \partial_x K_l \right\rangle_x v V_l(v, t) + (\omega_c \tau) \frac{Z_s}{A_s} \sum_k \left\langle X_i, E X_k \right\rangle_x \partial_v L_k(v, t) \\
&\quad = \nu_p \tau \sum_{s'} \nu_{ss'} \sum_k \left[ \frac{1}{m_s} \left\langle X_i, T_{ss'} X_k \right\rangle_x \partial_v^2 L_k(v, t) + \delta_{ik} \partial_v (v L_k(v, t)) - \left\langle X_i, u_{ss'} X_k \right\rangle_x \partial_v L_k(v, t) \right] 
\end{align*}

The boundary conditions on $K_j$ are the projection of the full distribution function BCs:
\begin{align}
    K_j^b(x_b) &= \langle V_j, \mathbf{1}_{\mathbf{n}(x_b) \cdot v < 0} f_s(x_b, v) \rangle_v \\
               &= \left\langle V_j, \mathbf{1}_{\mathbf{n}(x_b) \cdot v < 0} \left(\sum_{kl} X_k(x_b) S_{kl} V_l(v)\right) \right\rangle_v \\
               &= \sum_{l} K_l(x_b) \left\langle V_j, \mathbf{1}_{\mathbf{n}(x_b) \cdot v < 0} V_l(v) \right\rangle_v
\end{align}

\subsection{Spatial discretization}

\jack{
    \begin{itemize}
        \item I suggest using a finite volume scheme with MUSCL-like slope-limited reconstruction. Such an approach is extremely robust if using a TVD limiter, efficient, and easy to implement.
        \item For the spatial numerical flux, I suggest we use the kinetic flux vector splitting method applied to the DLR of the Boltzmann equation in \cite{huAdaptiveDynamicalLow2022}
        \item A similar numerical flux can be applied for the $E \partial_v f$ terms.
    \end{itemize}
}
\jack{}.

\printbibliography

\end{document}
